\documentclass[../body.tex]{subfiles}
\begin{document}
	 По сгенерированной согласно стандартному нормальному закону $N(x, 0, 1)$ выборке мы нашли интервалы, в которые с вероятностью $0.95$ попадут параметры закона, описывающего выборку.\\ То есть, имея выборку и зная про нее лишь то, что закон, по которому она сгенерирована, нормальный, смогли найти интервалы, в которых с заданной вероятностью лежат параметры неизвестного нам распределения.
	 Видим, что известные нам в этом эксперименте $m = 0, \sigma = 1$ действительно лежат в соответствующих найденных доверительных интервалах.\\
	 Заметим так же, что асимптотические оценки при увеличении мощности выборки приближаются к классическим.
	 Чем больше мощность выборки, тем более точными (т.е. меньшими по длине) являются доверительные интервалы.
\end{document}