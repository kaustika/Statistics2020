\documentclass[../body.tex]{subfiles}
\begin{document}
	\subsection{Двумерное нормальное распределение}
	Двумерная случайная величина $(X,Y)$ называется распределённой нормально (или просто нормальной), если её плотность вероятности определена формулой
	\begin{equation}
	N(x, y, \bar{x}, \bar{y}, \sigma_{x}, \sigma_{y}, \rho) = 
	\frac{1}{2\pi\sigma_{x}\sigma_{y}\sqrt{1-\rho^{2}}}
	exp{\begin{Bmatrix}
		\frac{-1}{2(1-\rho^{2})}
		\begin{bmatrix}
		\frac{(x-\bar{x})^{2}}{\sigma_{x}^{2}} - 2\rho\frac{(x-\bar{x})(y-\bar{y})}{\sigma_{x}\sigma_{y}} + \frac{(y-\bar{y})^{2}}{\sigma_{y}^{2}}
		\end{bmatrix}
		\end{Bmatrix}} \:(2)
	\end{equation}
	Компоненты X,Y двумерной нормальной случайной величины также распределены нормально с математическими ожиданиями $\bar{x}$,$\bar{y}$ и средними квадратическими отклонениями $\sigma_{x},\sigma_{y}$ соответственно \cite[c.~133-134]{max}.
	Параметр $\rho$ называется коэффициентом корреляции.
	
	
	
	\subsection{Корреляционный момент (ковариация) и коэффициент корреляции}
	Корреляционным моментом, иначе ковариацией, двух случайных величин $X$ и $Y$ называется математическое ожидание произведения отклонений этих случайных величин от их математических ожиданий \cite[c.~141]{max}.
	\begin{equation}
	K = cov(X, Y) = M[(X - \bar{x})(Y - \bar{y})]
	\label{K}
	\eqno(3)
	\end{equation}
	Коэффициентом корреляции $\rho$ двух случайных величин $X$ и $Y$ называется отношение их корреляционного момента к произведению их средних квадратических отклонений:
	\begin{equation}
	\rho = \frac{K}{\sigma_{x}\sigma_{y}}
	\label{ro}
	\eqno(4)
	\end{equation}
	Коэффициент корреляции — это нормированная числовая характеристика, являющаяся мерой близости зависимости между случайными величинами к линейной \cite[c.~150]{max}.
	
	\subsection{Выборочные коэффициенты корреляции}
	\subsubsection{Выборочный коэффициент корреляции Пирсона}
	Пусть по выборке значений ${x_{i},y_{i}}^{n}_{1}$ двумерной с.в. $(X,Y )$ требуется оценить коэффициент корреляции $\rho = \frac{cov(X,Y)}{\sqrt{DXDY}}$ . Естественной оценкой для $\rho$ служит его статистический аналог в виде выборочного коэффициента корреляции, предложенного К.Пирсоном, —
	\begin{equation}
	r = \frac{
		\frac{1}{n}\sum{(x_{i} - \bar{x})(y_{i}-\bar{y})}
	}{
		\sqrt{\frac{1}{n}\sum{(x_{i} - \bar{x})^{2}}\frac{1}{n}\sum{(y_{i} - \bar{y})^{2}}}
	}=\frac{K}{s_{X}s_{Y}},
	\label{r}
	\eqno(5)
	\end{equation}
	где $K,s^{2}_{X},s^{2}_{Y}$ — выборочные ковариация и дисперсии с.в. X и Y \cite[c.~535]{max}.
	
	
	\subsubsection{Выборочный квадрантный коэффициент корреляции}
	Кроме выборочного коэффициента корреляции Пирсона, существуют и другие оценки степени взаимосвязи между случайными величинами. К ним относится выборочный квадрантный коэффициент корреляции
	\begin{equation}
	r_{Q} = \frac{(n_{1} + n_{3}) - (n_{2} + n_{4})}{n},
	\label{rQ}
	\eqno(6)
	\end{equation}
	где $n_{1}, n_{2},n_{3}$ и $n_{4}$ — количества точке с координатами $(x_{i},y_{i})$, попавшими соответственно в \RomanNumeralCaps{1}, \RomanNumeralCaps{2}, \RomanNumeralCaps{3}, \RomanNumeralCaps{4}  квадранты декартовой системы с осями $x′ = x-med x, y′ = y- med y$  и с центром в точке с координатами $(med x,med y)$ \cite[c.~539]{max}.
	
	
	
	\subsubsection{Выборочный коэффициент ранговой корреляции Спирмена}
	На практике нередко требуется оценить степень взаимодействия между качественными признаками изучаемого объекта. Качественным называется признак, который нельзя измерить точно, но который позволяет сравнивать изучаемые объекты между собой и располагать их в порядке убывания или возрастания их качества. Для этого объекты выстраиваются в определённом порядке в соответствии с рассматриваемым признаком. Процесс упорядочения называется ранжированием, и каждому члену упорядоченной последовательности объектов присваивается ранг, или порядковый номер. Например, объекту с наименьшим значением признака присваивается ранг $1$, следующему за ним объекту — ранг $2$, и т.д. Таким образом, происходит сравнение каждого объекта со всеми объектами изучаемой выборки.
	\newline
	Если объект обладает не одним, а двумя качественными признаками — переменными $X$ и $Y$ , то для исследования их взаимосвязи используют выборочный коэффициент корреляции между двумя последовательностями рангов этих признаков.
	\newline
	Обозначим ранги, соотвествующие значениям переменной $X$, через u, а ранги, соотвествующие значениям переменной $Y$, — через $v$.
	\newline
	Выборочный коэффициент ранговой корреляции Спирмена определяется как выборочный коэффициент корреляции Пирсона между рангами $u,v$ переменных $X,Y$ :
	\begin{equation}
	r_{S} = \frac{
		\frac{1}{n}\sum{(u_{i} - \bar{u})(v_{i}-\bar{v})}
	}{
		\sqrt{\frac{1}{n}\sum{(u_{i} - \bar{u})^{2}}\frac{1}{n}\sum{(v_{i} - \bar{v})^{2}}}
	},
	\label{rS}
	\eqno(7)
	\end{equation}
	где $\bar{u} = \bar{v} = \frac{1 + 2 + ... + n}{n} = \frac{n + 1}{2}$ — среднее значение рангов \cite[c.~540-541]{max}.
	
	
	\subsection{Эллипсы рассеивания}
	Рассмотрим поверхность распределения, изображающую функцию (1). Она имеет вид холма, вершина которого находится над точкой $(\bar{x},\bar{y})$.
	\newline
	В сечении поверхности распределения плоскостями, параллельными оси \\$N(x, y, \bar{x}, \bar{y}, \sigma_{x}, \sigma_{y}, \rho)$, получаются кривые, подобные нормальным кривым распределения. В сечении поверхности распределения плоскостями, параллельными плоскости $xOy$, получаются эллипсы. Напишем уравнение проекции такого эллипса на плоскость $xOy$: 
	\begin{equation}
	\frac{(x-\bar{x})^{2}}{\sigma_{x}^{2}} - 
	2\rho\frac{(x-\bar{x})(y-\bar{y})}{\sigma_{x}\sigma_{y}}+
	\frac{(y-\bar{y})^{2}}{\sigma_{y}^{2}} = const
	\label{ellipse}
	\eqno(8)
	\end{equation}
	Уравнение эллипса (8) можно проанализировать обычными методами аналитической геометрии. Применяя их, убеждаемся, что центр эллипса (8) находится в точке с координатами $(\bar{x},\bar{y})$; что касается направления осей симметрии эллипса, то они составляют с осью Ox углы, определяемые уравнением
	\begin{equation}
	tg(2\alpha) = \frac{2\rho\sigma_{x}\sigma_{y}}{\sigma_{x}^{2} - \sigma_{y}^{2}}
	\label{angle}
	\eqno(9)
	\end{equation}
	Это уравнение дает два значения углов: $\alpha$ и $\alpha_{1}$, различающиеся на $\frac{\pi}{2}$.
	\newline
	Таким образом, ориентация эллипса (8) относительно координатных осей находится в прямой зависимости от коэффициента корреляции $\rho$ системы $(X,Y)$; если величины не коррелированны (т.е. в данном случае и независимы), то оси симметрии эллипса параллельны координатным осям; в противном случае они составляют с координатными осями некоторый угол.
	\newline
	Пересекая поверхность распределения плоскостями, параллельными плоскости $xOy$, и проектируя сечения на плоскость $xOy$ мы получим целое семейство подобных и одинаково расположенных эллипсов с общим центром $(\bar{x},\bar{y})$. Во всех точках каждого из таких эллипсов плотность распределения $N(x, y, \bar{x}, \bar{y}, \sigma_{x}, \sigma_{y}, \rho)$ постоянна. Поэтому такие эллипсы называются эллипсами равной плотности или, короче эллипсами рассеивания. Общие оси всех эллипсов рассеивания называются главными осями рассеивания \cite[c.~193-194]{regr}.
	
\end{document}