\documentclass[../body.tex]{subfiles}
\begin{document}
	\subsection{Задание}
	Сгенерировать двумерные выборки размерами $20, 60, 100$ для нормального двумерного распределения $N(x,y,0,0,1,1,\rho)$. Коэффициент корреляции $\rho$ взять равным $0, 0.5, 0.9.$ Каждая выборка генерируется $1000$ раз и для неё вычисляются: среднее значение, среднее значение квадрата и дисперсия коэффициентов корреляции Пирсона, Спирмена и квадрантного коэффициента корреляции. Повторить все вычисления для смеси нормальных распределений:
	\begin{equation}
	f(x,y) = 0.9N(x,y,0,0,1,1,0.9) + 0.1N(x,y,0,0,10,10,−0.9) \eqno(1)
	\end{equation}
	Изобразить сгенерированные точки на плоскости и нарисовать эллипс равновероятности.
\end{document}

