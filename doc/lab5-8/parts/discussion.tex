\documentclass[../body.tex]{subfiles}
\begin{document}
	\subsection{Выборочные коэффициенты корреляции и эллипсы рассеяния}
	\begin{itemize}
		\item Для двумерного нормального распределения дисперсии выборочных коэффициентов корреляции упорядочены следующим образом: $r < r_{S} < r_{Q}$; для смеси распределений: $r_{Q} < r_{S} < r$.
		\item Процент попавших элементов выборки в эллипс рассеивания ($95\%$-ная доверительная область) примерно равен его теоретическому значению ($95\%$).
	\end{itemize}
	
	\subsection{Оценки коэффициентов линейной регрессии} Оценим, какие коэффициенты, полученные каким методом лучше аппроксимируют модельную зависимость: для МНК и МНМ для выборки с возмущениями и без них посчитаем сумму по всем $x \in [-1.8, 2]$, взятым с шагом $0.2$: $$Distance_{ls(lm)} = \sum{(y_{model} - y_{ls(lm)})^2}.$$
	$$y_{model} = 2 + 2 \cdot x$$
	$$y_{ls} = \hat{a}_{ls} + \hat{b}_{ls} \cdot x$$
	$$y_{ls} = \hat{a}_{lm} + \hat{b}_{lm} \cdot x$$
	Для данных выборок получаем:
	\begin{itemize}
		\item Выборка без возмущений: $$Distance_{ls} < Distance_{lm}$$ $$0.563  <  1.637$$
		Критерий наименьших квадратов точнее оценивает коэффициенты линейной регрессии на выборке без возмущений. 
		\item Выборка без возмущений: $$Distance_{lm} < Distance_{ls}$$ $$1.393  <  43.805$$
		Для выборки с возмущениями результат получается точнее при оценке критерием наименьших модулей.
	\end{itemize}
	Таким образом, критерий наименьших модулей устойчив к редким выбросам, в отличие от критерия наименьших квадратов, что соответствует ожиданиям, ведь он обладает робастными свойствами.
	
	\subsection{Проверка гипотезы о законе распределения генеральной совокупности. Метод хи-квадрат}
	По результатам проверки на близость с помощью критерия хи-квадрат можно принять гипотезу $H_0$ о нормальном распределении $N(x, \hat{\mu}, \hat{\sigma})$ на уровне значимости $ \alpha = 0.05$ для выборки, сгенерированной согласно $N(x, 0, 1)$. То есть, если взять в качестве гипотезы нормальное распределение с параметрами сдвига и масштаба равными оценкам максимального правдоподобия для $\mu, \sigma$, вычисленным по выборке $\sim N(x, 0, 1)$, то критерий отразит эту согласованность.\\
	Видим так же, что критерий принял гипотезу о том, что $20$-элементная выборка, сгенерированная согласно $N(x, 0, 1)$, описывается законом распределения $Laplace(x, \hat{\mu},\frac{\hat{\sigma}}{\sqrt2} )$(тут опять $\hat{\mu}, \hat{\sigma}$ - оценки максимального правдоподобия для $\mu, \sigma$, вычисленным по той же $20$-элементной стандартной нормальной выборке.)\\То есть, при малых мощностях выборки критерий хи-квадрат не почувствовал разницы между нормально распределенной случайной величиной и распределенной по Лапласу. Это ожидаемый результат, ведь выборка довольно мала, законы схожи по форме и параметры масштаба и сдвига выбраны тоже так, чтобы законы максимально друг к другу приблизить. 
	
	\subsection{Доверительные интервалы для параметров распределения}
	По сгенерированной согласно стандартному нормальному закону $N(x, 0, 1)$ выборке мы нашли интервалы, в которые с вероятностью $0.95$ попадут параметры закона, описывающего выборку.\\ То есть, имея выборку и зная про нее лишь то, что закон, по которому она сгенерирована, нормальный, смогли найти интервалы, в которых с заданной вероятностью лежат параметры неизвестного нам распределения.
	Видим, что известные нам в этом эксперименте $m = 0, \sigma = 1$ действительно лежат в соответствующих найденных доверительных интервалах.\\
	Заметим так же, что асимптотические оценки при увеличении мощности выборки приближаются к классическим.
	Чем больше мощность выборки, тем более точными (т.е. меньшими по длине) являются доверительные интервалы.
\end{document}