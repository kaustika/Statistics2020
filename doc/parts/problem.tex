\documentclass[../body.tex]{subfiles}
\begin{document}
Для 5 распределений:
\begin{enumerate}
	\item $N(x, 0, 1)$ -- нормальное распределение
	\item $C(x, 0, 1)$ -- распределение Коши
	\item $L(x, 0, \frac{1}{\sqrt{2}})$ -- распределение Лапласа 
	\item $P(k, 10)$ -- распределение Пуассона
	\item $U(x, -\sqrt{3}, \sqrt{3})$ -- расномерное распределение
\end{enumerate}

\subsection{Задание 1}
Сгенерировать выборки размером 10, 50 и 1000 элементов.\newline Построить на одном рисунке гистограмму и график плотности распределения.
\subsection{Задание 2}
Сгенерировать выборки размером 10, 100 и 1000 элементов.
Для каждой выборки вычислить следующие статистические характеристики положения данных: $\overline{x}, med x, z_R, z_Q, z_{tr}.$ Повторить такие вычисления 1000 раз для каждой выборки и найти среднее характеристик положения и их квадратов:
\begin{equation}
	E(z) = \overline{z} \eqno(1)
\end{equation}
Вычислить оценку дисперсии по формуле:
\begin{equation}
	D(z) = \overline{z^2} - \overline{z}^2 \eqno(2)
\end{equation}
Представить полученные данные в виде таблиц.
\subsection{Задание 3}
Сгенерировать выборки размером 20 и 100 элементов.
Построить для них боксплот Тьюки.
Для каждого распределения определить долю выбросов экспериментально (сгенерировав выборку, соответствующую распределению 1000
раз, и вычислив среднюю долю выбросов) и сравнить с результатами, полученными теоретически.


\end{document}

