\documentclass[../body.tex]{subfiles}
\begin{document}
	\subsection{Гистограмма и график плотности распределения}
	По результатам проделанной работы можем сделать вывод о том, что чем больше выборка для каждого из распределений, тем ближе ее гистограмма к графику плотности вероятности того закона, по котророму распределены величины сгенерированной выборки. Чем меньше выборка, тем менее она показательна - тем хуже по ней определяется характер распределения величины.
	\subsection{Характеристики положения и рассеяния}
	Заметим, что для распределения Коши $D(z)$ - быстро растет с ростом размера выборки, по которой берется $z_R$ или $\overline{x}$, что отличает его от других типов распределений, для которых дисперсия характеристик рассеяния убывает с увеличением размера выборки - это происходит из-за характерных для распределения Коши выбросов, которые мы уже могли заметить на гистограммах в Задании 1.
	\subsection{Доля и теоретическая вероятность выбросов}
	Сравним долю выбросов определенную экспериментально с результатами, полученными теоретически. Видим точное соответствие с теорией для равномерного распределения - вероятность нулевая и выбросов мы не получили.\newline
	Результаты для выборок, сгенерированных в соответствии с законами распределения Лапласа и Коши, оказались близкими к теории, а доля выбросов для распределений Пуассона и Нормального ниже соответствующих теоретических оценок.\newline
	Заметим, что все распределения дают для большей выборки (100 элементов) результат ближе к теории, чем для меньшей выборки (20 элементов).
\end{document}