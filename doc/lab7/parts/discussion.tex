\documentclass[../body.tex]{subfiles}
\begin{document}
	По результатам проверки на близость с помощью критерия хи-квадрат можно принять гипотезу $H_0$ о нормальном распределении $N(x, \hat{\mu}, \hat{\sigma})$ на уровне значимости $ \alpha = 0.05$ для выборки, сгенерированной согласно $N(x, 0, 1)$. То есть, если взять в качестве гипотезы нормальное распределение с параметрами сдвига и масштаба равными оценкам максимального правдоподобия для $\mu, \sigma$, вычисленным по выборке $\sim N(x, 0, 1)$, то критерий отразит эту согласованность.\\
	Видим так же, что критерий принял гипотезу о том, что $20$-элементная выборка, сгенерированная согласно $N(x, 0, 1)$, описывается законом распределения $Laplace(x, \hat{\mu},\frac{\hat{\sigma}}{\sqrt2} )$(тут опять $\hat{\mu}, \hat{\sigma}$ - оценки максимального правдоподобия для $\mu, \sigma$, вычисленным по той же $20$-элементной стандартной нормальной выборке.)\\То есть, при малых мощностях выборки критерий хи-квадрат не почувствовал разницы между нормально распределенной случайной величиной и распределенной по Лапласу. Это ожидаемый результат, ведь выборка довольно мала, законы схожи по форме и параметры масштаба и сдвига выбраны тоже так, чтобы законы максимально друг к другу приблизить. 
\end{document}