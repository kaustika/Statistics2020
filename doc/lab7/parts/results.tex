\documentclass[../body.tex]{subfiles}
\begin{document}
	\begin{enumerate}
		\item \textbf{Нормальная выборка, нормальная гипотеза $H_0$.}\\
		Для нормального распределения согласно выведенным в примере(\ref{normal}) оценкам максимального правдоподобия:
		\begin{itemize}
			\item $\hat{\mu} = \overline{x} $
			\item $\hat{\sigma}^2 = s^2 = \frac{1}{n}\sum_{i = 1}^{n}(x_i - \overline{x})^2$
		\end{itemize}
	Тогда $\hat{\mu} \approx 0.07, \hat{\sigma} \approx 0.9$.\\ И проверяем гопотезу $H_0$, что сгенерированная по закону $N(x, 0, 1)$ $100$-элементная выборка подчиняется закону $N(x, \hat{\mu}, \hat{\sigma})$.
	Используем критерий согласия $\chi^2$:
	\begin{itemize}
		\item $\alpha = 0.05$ - уровень значимости,
		\item $n = 100$ - размер выборки,
		\item $k := \lfloor1+3.3\lg 100\rfloor = \lfloor7.6\rfloor = 7$ - количество промежутков,
		\item Квантиль $\chi_{1 - \alpha}^2(k - 1) = \chi_{0.95}^2(6) = //$ из таблицы\cite[с.~358]{math}$// \approx 12.59$
	\end{itemize}

	\begin{table}[H]
		\centering
		\begin{tabular}{| c | c | c | c | c | c | c |}
			\hline \hline
			&  &  &  &  &  & \\
			$i$   & $\Delta_i = [a_{i-1}, a_i)$   &   $n_i$ &   $p_i$ &   $np_i$ &   $n_i - np_i$ &   $(n_i - np_i)^2/(np_i)$ \\
			&  &  &  &  &  & \\
			\hline
			&  &  &  &  &  & \\
			1     & ($-\infty$, -1.5)           &       3 &  0.0401 &     4.01 &          -1.01 &                            0.25 \\ 
			&  &  &  &  &  & \\\hline &  &  &  &  &  & \\
			2     & [-1.5, -0.9)                &      11 &  0.0996 &     9.96 &           1.04 &                            0.11 \\ 
			&  &  &  &  &  & \\\hline &  &  &  &  &  & \\
			3     & [-0.9, -0.3)                &      16 &  0.1998 &    19.98 &          -3.98 &                            0.79 \\ 
			&  &  &  &  &  & \\\hline &  &  &  &  &  & \\
			4     & [-0.3, 0.3)                 &      33 &  0.2608 &    26.08 &           6.92 &                            1.84 \\ 
			&  &  &  &  &  & \\\hline &  &  &  &  &  & \\
			5     & [0.3, 0.9)                  &      20 &  0.2215 &    22.15 &          -2.15 &                            0.21 \\ 
			&  &  &  &  &  & \\\hline &  &  &  &  &  & \\
			6     & [0.9, 1.5)                  &      10 &  0.1224 &    12.24 &          -2.24 &                            0.41 \\ 
			&  &  &  &  &  & \\\hline &  &  &  &  &  & \\
			7     & [1.5, $+\infty$)             &       7 &  0.0559 &     5.59 &           1.41 &                            0.35 \\ 
			&  &  &  &  &  & \\\hline &  &  &  &  &  & \\
			$\sum$& ($-\infty$, $+\infty$)                           &     100 &  1      &   100    &           0    &                            3.96 = $\chi_B^2$ \\
			&  &  &  &  &  & \\\hline \hline
		\end{tabular}
		\caption{Вычисление $\chi_B^2$ при проверке гипотезы $H_0$ о нормальном законе распределения $N(x, \hat{\mu}, \hat{\sigma})$.}
		\label{chi2_normal}
	\end{table}
Сравним $\chi_{0.95}^2(6) \approx 12.59$ и найденное  $\chi_B^2 \approx 3.96: 12.59 > 3.96.$ Следовательно гипотезу $H_0$ на данном этапе проверки можно принять.\\
\item \textbf{Нормальная выборка, гипотеза Лапласа}\\
Рассмотрим гипотезу $H_0^*$, что выборка распределена согласно закону $Laplace(x, \hat{\mu},\frac{\hat{\sigma}}{\sqrt2} )$
Используем критерий согласия $\chi^2$:
\begin{itemize}
	\item $\alpha = 0.05$ - уровень значимости,
	\item $n = 20$ - размер выборки,
	\item $k := \lfloor1+3.3\lg 20\rfloor = \lfloor5.3\rfloor = 5$ - количество промежутков,
	\item Квантиль $\chi_{1 - \alpha}^2(k - 1) = \chi_{0.95}^2(4) = //$ из таблицы\cite[с.~358]{math}$// \approx 9.49$
\end{itemize}
\begin{table}[H]
	\centering
	\begin{tabular}{| c | c | c | c | c | c | c |}
		\hline \hline
		&  &  &  &  &  & \\
		$i$   & $\Delta_i = [a_{i-1}, a_i)$   &   $n_i$ &   $p_i$ &   $np_i$ &   $n_i - np_i$ &   $(n_i - np_i)^2/np_i$ \\
		&  &  &  &  &  & \\
		\hline
		&  &  &  &  &  & \\
		1     & ($-\infty$, -1.5]           &       1 &  0.0696 &     1.39 &          -0.39 &                    0.11 \\
		&  &  &  &  &  & \\\hline &  &  &  &  &  & \\
		2     & [-1.5, -0.5)                  &       8 &  0.2499 &     5    &           3    &                    1.8  \\
		&  &  &  &  &  & \\\hline &  &  &  &  &  & \\
		3     & [-0.5, 0.5)                   &       6 &  0.5101 &    10.2  &          -4.2  &                    1.73 \\
		&  &  &  &  &  & \\\hline &  &  &  &  &  & \\
		4     & [0.5, 1.)                     &       5 &  0.1333 &     2.67 &           2.33 &                    2.04 \\
		&  &  &  &  &  & \\\hline &  &  &  &  &  & \\
		5     & [1.5, $+\infty$)            &       0 &  0.0371 &     0.74 &          -0.74 &                    0.74 \\
		&  &  &  &  &  & \\\hline &  &  &  &  &  & \\
		$\sum$  & ($-\infty$, $+\infty$)                          &      20 &  1      &    20    &           0    &                    6.43 = $\chi_B^2$ \\
		&  &  &  &  &  & \\\hline \hline
	\end{tabular}
	\caption{Вычисление $\chi_B^2$ при проверке гипотезы $H_0^*$ о $Laplace(x, \hat{\mu},\frac{\hat{\sigma}}{\sqrt2} )$.}
	\label{chi2_laplace}
\end{table}
Сравним $\chi_{0.95}^2(4) \approx 9.49$ и найденное  $\chi_B^2 \approx 6.43: 9.49 > 6.43.$ Следовательно гипотезу $H_0^*$ на данном этапе проверки можно принять.\\


\end{enumerate}

\end{document}